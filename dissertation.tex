% The document class supplies options to control rendering of some standard
% features in the result.  The goal is for uniform style, so some attention 
% to detail is *vital* with all fields.  Each field (i.e., text inside the
% curly braces below, so the MEng text inside {MEng} for instance) should 
% take into account the following:
%
% - author name       should be formatted as "FirstName LastName"
%   (not "Initial LastName" for example),
% - supervisor name   should be formatted as "Title FirstName LastName"
%   (where Title is "Dr." or "Prof." for example),
% - degree programme  should be "BSc", "MEng", "MSci", "MSc" or "PhD",
% - dissertation title should be correctly capitalised (plus you can have
%   an optional sub-title if appropriate, or leave this field blank),
% - dissertation type should be formatted as one of the following:
%   * for the MEng degree programme either "enterprise" or "research" to
%     reflect the stream,
%   * for the MSc  degree programme "$X/Y/Z$" for a project deemed to be
%     X%, Y% and Z% of type I, II and III.
% - year              should be formatted as a 4-digit year of submission
%   (so 2014 rather than the accademic year, say 2013/14 say).

\documentclass[ % the name of the author
                    author={Tim Nguyen},
                % the name of the supervisor
                supervisor={Dr. David Bernhard},
                % the degree programme
                    degree={MEng},
                % the dissertation    title (which cannot be blank)
                     title={Implementing CSS conic gradients in the Firefox rendering engine},
                % the dissertation subtitle (which can    be blank)
                  subtitle={},
                % the dissertation     type
                      type={enterprise},
                % the year of submission
                      year={2020} ]{dissertation}

\begin{document}


% =============================================================================

% This macro creates the standard UoB title page by using information drawn
% from the document class (meaning it is vital you select the correct degree 
% title and so on).

\maketitle

% After the title page (which is a special case in that it is not numbered)
% comes the front matter or preliminaries; this macro signals the start of
% such content, meaning the pages are numbered with Roman numerals.

\frontmatter

% This macro creates the standard UoB declaration; on the printed hard-copy,
% this must be physically signed by the author in the space indicated.

\makedecl

% LaTeX automatically generates a table of contents, plus associated lists 
% of figures, tables and algorithms.  The former is a compulsory part of the
% dissertation, but if you do not require the latter they can be suppressed
% by simply commenting out the associated macro.

\tableofcontents
\listoffigures
\listoftables
\listofalgorithms
\lstlistoflistings

% The following sections are part of the front matter, but are not generated
% automatically by LaTeX; the use of \chapter* means they are not numbered.

% -----------------------------------------------------------------------------

\chapter*{Executive Summary}

{\bf A compulsory section, of at most $1$ page} 
\vspace{1cm}

\noindent
Firefox is the second most popular web browser, it is open source and
developed by Mozilla. This project aims to implement CSS conic gradients
into the Gecko, the Firefox rendering engine. Firefox was the only web
browser as of January 2020 that did not support this feature.

For this project, \texttt{conic-gradient} has been implemented for most platforms (macOS, Linux, some Windows devices and Android).

% This section should pr\'{e}cis the project context, aims and objectives,
% and main contributions (e.g., deliverables) and achievements; the same 
% section may be called an abstract elsewhere.  The goal is to ensure the 
% reader is clear about what the topic is, what you have done within this 
% topic, {\em and} what your view of the outcome is.

% \noindent
% The latter aspects should (ideally) be presented as a concise, factual 
% bullet point list.  Again the points will differ for each project, but 
% an might be as follows:

\begin{quote}
\noindent
\begin{itemize}
\item I spent $120$ hours collecting material on and learning about the 
      Java garbage-collection sub-system. 
\item I wrote a total of $5000$ lines of source code, comprising a Linux 
      device driver for a robot (in C) and a GUI (in Java) that is 
      used to control it.
\item I designed a new algorithm for computing the non-linear mapping 
      from A-space to B-space using a genetic algorithm, see page $17$.
\item I implemented a version of the algorithm proposed by Jones and 
      Smith in [6], see page $12$, corrected a mistake in it, and 
      compared the results with several alternatives.
\end{itemize}
\end{quote}

% -----------------------------------------------------------------------------
care of rendering.


\chapter*{Supporting Technologies}

{\bf A compulsory section, of at most $1$ page}
\vspace{1cm} 

\noindent
This section should present a detailed summary, in bullet point form, 
of any third-party resources (e.g., hardware and software components) 
used during the project.  Use of such resources is always perfectly 
acceptable: the goal of this section is simply to be clear about how
and where they are used, so that a clear assessment of your work can
result.  The content can focus on the project topic itself (rather,
for example, than including ``I used \mbox{\LaTeX} to prepare my 
dissertation''); an example is as follows:

\begin{quote}
\noindent
\begin{itemize}
\item I used the Java {\tt BigInteger} class to support my implementation 
      of RSA.
\item I used a parts of the OpenCV computer vision library to capture 
      images from a camera, and for various standard operations (e.g., 
      threshold, edge detection).
\item I used an FPGA device supplied by the Department, and altered it 
      to support an open-source UART core obtained from 
      \url{http://opencores.org/}.
\item The web-interface component of my system was implemented by 
      extending the open-source WordPress software available from
      \url{http://wordpress.org/}.
\end{itemize}
\end{quote}

% -----------------------------------------------------------------------------

\chapter*{Acknowledgements}

{\bf An optional section, of at most $1$ page}
\vspace{1cm} 

\noindent
Credits go to the following people for making this project a success:
\begin{itemize}
\item The Firefox Layout Team in particular to Jonathan Watt and Sean Voisen for finding this project
\item The Firefox Graphics Team for WebRender code reviews and answering questions related to WebRender

\item Dr. David Bernhard for supervising this project
\item Emilio Cobos Álvarez for most code reviews and answering questions related to the Firefox Style System
\item Lee Salzman for Skia code reviews
\item Markus Stange and Matt Woodrow for web painting code reviews
\end{itemize}
% =============================================================================

% After the front matter comes a number of chapters; under each chapter,
% sections, subsections and even subsubsections are permissible.  The
% pages in this part are numbered with Arabic numerals.  Note that:
%
% - A reference point can be marked using \label{XXX}, and then later
%   referred to via \ref{XXX}; for example Chapter\ref{chap:context}.
% - The chapters are presented here in one file; this can become hard
%   to manage.  An alternative is to save the content in seprate files
%   the use \input{XXX} to import it, which acts like the #include
%   directive in C.

\mainmatter

% -----------------------------------------------------------------------------

\chapter{Background}
\label{chap:context}

\vspace{1cm} 

\noindent

\section{Web Standards}


Mozilla Firefox uses its own rendering engine. The rendering engine takes care
care of rendering.

\section{CSS}
 is a programming language that used to define styling (background images, text colors, etc.) for documents written in HTML, XHTML, SVG or other markup languages typically used for web pages. 



\section{Web standards}


\section{Web compatibility}





% This chapter should describe the project context, and motivate each of
% the proposed aims and objectives.  Ideally, it is written at a fairly 
% high-level, and easily understood by a reader who is technically 
% competent but not an expert in the topic itself.

% In short, the goal is to answer three questions for the reader.  First, 
% what is the project topic, or problem being investigated?  Second, why 
% is the topic important, or rather why should the reader care about it?  
% For example, why there is a need for this project (e.g., lack of similar 
% software or deficiency in existing software), who will benefit from the 
% project and in what way (e.g., end-users, or software developers) what 
% work does the project build on and why is the selected approach either
% important and/or interesting (e.g., fills a gap in literature, applies
% results from another field to a new problem).  Finally, what are the 
% central challenges involved and why are they significant? 
 
% The chapter should conclude with a concise bullet point list that 
% summarises the aims and objectives.  For example:

\begin{quote}
\noindent
The high-level objective of this project is to reduce the performance 
gap between hardware and software implementations of modular arithmetic.  
More specifically, the concrete aims are:

\begin{enumerate}
\item Research and survey literature on public-key cryptography and
      identify the state of the art in exponentiation algorithms.
\item Improve the state of the art algorithm so that it can be used
      in an effective and flexible way on constrained devices.
\item Implement a framework for describing exponentiation algorithms
      and populate it with suitable examples from the literature on 
      an ARM7 platform.
\item Use the framework to perform a study of algorithm performance
      in terms of time and space, and show the proposed improvements
      are worthwhile.
\end{enumerate}
\end{quote}


% -----------------------------------------------------------------------------

\chapter{The Firefox Development process}
\label{chap:process}

\noindent
This chapter describes how to set up a development environment to build Firefox
and the Firefox development process in general. It is worth noting that due to 
the 20-year old history of the source code, there are different workflows that 
have been developed and documented which all work today. This section will
describe the workflow that I have used for development, which should be close
to the latest recommended one.

\section{Finding issues to work on}

While some of Mozilla's projects are on Github, most older projects like
Firefox are tracked on Mozilla's own issue tracker called
\href{https://bugzilla.mozilla.org}{Bugzilla}.

To find simple issues to get familiar with the development process, there
are "Good first bugs" which can be browsed on \href{https://codetribute.mozilla.org/}{Codetribute}.
These issues usually have mentors assigned to them and are simple changes 
with step-by-step instructions.


\section{Getting the source code}
Firefox development uses Mercurial (hg), a version-control system similar to Git,
although Git can also be used through ports or mirrors.

The command mentioned in the documentation to clone the Firefox source code is:

\begin{verbatim}
hg clone --uncompressed https://hg.mozilla.org/mozilla-unified
\end{verbatim}

mozilla-unified is the repository combining the commit history of all other Firefox
repositories like mozilla-central, mozilla-beta or mozilla-release (which will be
described at the end of this section) as different branches.

The command above usually takes about 30 minutes depending on the network connection.
Once the cloning is done, the central branch should be checked out, since it contains
the latest stable changes:

\begin{verbatim}
hg up central
\end{verbatim}

\texttt{hg up} is a shortcut for \texttt{hg update}, which is equivalent to the
\texttt{git checkout} command on Git, as it lets you check out the source code at
different commits.

\section{Installing dependencies}

Once the source code is cloned, the dependencies can be installed by running the following command:

\begin{verbatim}
./mach bootstrap
\end{verbatim}

This will bring up an interactive wizard that with options to pick from to install
the tools appropriate for the developer. The wizard will ask whether to use full or artifact builds
at some point. Full builds are used for C++ and Rust changes, since they require the full binaries
to be regenerated, while artifact builds are typically used for front-end changes where binaries
can be downloaded from a server. For this project, full builds are necessary since it involves changes
on the rendering engine written in C++ and Rust.

\noindent On Windows, extra steps need to be done beforehand, quoted from the documentation:

\begin{quote}
    1. You need 64-bit version of Windows 7 or later.
    
    2. Download and install Visual Studio.
    
    3. Finally download the MozillaBuild Package. Installation directory should be: c:\textbackslash\textbackslash mozilla-build\textbackslash
\end{quote}

\section{Building Firefox}

Once the changes have been made to the source code, in order to test the changes,
building Firefox is needed. This is done using the following command:

\begin{verbatim}
./mach build
\end{verbatim}

The first time, this command takes about one hour for full builds, while artifact builds
take about 5 minutes depending on the network connection. Once finished, the build can
be ran using:

\begin{verbatim}
./mach run
\end{verbatim}

\section{Getting around the code}

Mozilla provides a very useful search tool where code is regularly indexed to provide



\section{Committing the changes}

Once the changes have been are tested, they should be committed to be sent for review.
This is done using the command below:

\begin{verbatim}
hg commit -m "Bug XXX - Commit message. r=reviewer"
\end{verbatim}

\noindent where:
\begin{itemize}
\item \texttt{XXX} is the issue number on Bugzilla.
\item "Commit message" is a sentence summarising the changes.
\item the reviewer is referred by their nickname.
\end{itemize}

Subsequent changes to the same revision can be done by amending the commit:
\begin{verbatim}
hg commit --amend
\end{verbatim}

If the commit is out of date, it is possible to update it by pulling the latest
source code using \texttt{hg pull}, then by rebasing the commit on top of the
central branch using \texttt{hg rebase -d central}.

\section{Submitting for review}

Mozilla uses a third-party tool called \href{https://phacility.com/phabricator/}{Phabricator}
for code reviews. Mozilla's instance of Phabricator is integrated with a set of
in-house tools:

\begin{itemize}
    \item \texttt{moz-phab}: a command-line utility to publish commits to Phabricator
    \item Bugzilla integration bot, linking Bugzilla issues with Phabricator revisions
    \item A code review bot that runs on every submitted revision and adding automated
review comments from linters.
    \item Lando: a tool that allows merging the revision once it is accepted.
\end{itemize}

In order to submit a commit for review, it is necessary to have a Bugzilla account
to login to Phabricator. Once logged into Phabricator, assuming the commit to be submitted
is already checked out, submitting for review can be done using the \texttt{moz-phab submit} command.

The commit is now submitted for review. Subsequent changes to address comments from
reviewers should be amended to the commit using \texttt{hg commit --amend} and re-submitted using
\texttt{moz-phab submit}.

\section{Testing}

To ensure changes work correctly and don't break existing features, it is necessary to perform
testing on the software. Manual testing can be done by the developer by running the compiled build of Firefox.
However, for complex software like Firefox, there are a lot of tasks that need to be tested, 
so manually testing would take a large amount of time. This is where automated tests become useful.

These are many types of tests in Firefox, but for the sake of simplicity, only the ones relevant to
this project will be described.

\subsection{Mochitests}

...

\subsection{Reference testing}

...

\subsection{Web Platform Tests}

For complex changes, it is desirable to run the complete test suite to check if the change
breaks anything. While it is possible to do this on locally, it takes a large amount of time
and makes the computer not usable since some tests require the test window to be stay focused
all the time. to use Mozilla provides a "Try" server where commits can be pushed as often as wanted

\section{Landing the change}

...

mozilla-central is the repository that contains the latest source code that's built
twice a day by Mozilla's integration server into \href{https://nightly.mozilla.org/}{Firefox Nightly}, 
the alpha version of Firefox, equivalent to Google Chrome Canary.

% -----------------------------------------------------------------------------

\chapter{Technical background}
\label{chap:technical}

\section{Anatomy of a web page}
Web pages are documents written using markup languages like HTML. Their styling is
defined using the Cascading Style Sheets (CSS) programming language. 

TODO: quick description of what HTML is.

A CSS file contains a set of CSS rules, which themselves contain a CSS selector defining which
elements should be styled, and a set of CSS properties and values which define how those elements should be styled.

For instance, the following CSS rule gives all paragraphs white text and a blue background:
\begin{verbatim}
p {
    background-color: blue;
    color: white;
}
\end{verbatim}

\section{Anatomy of a browser engine}
In order to describe the implementation, it is useful to first provide an in-breadth overview of
how a browser rendering engine renders web pages. This happens in 5 main steps, which are roughly
similar for all 3 major rendering engines.

The first step is parsing, HTML and CSS files are parsed into data structures that are understood
by the rendering engine. For this project, only CSS parsing is relevant.

The second step is styling. The CSS engine figures out which CSS properties should apply to each
element on the web page, for which the process isn't relevant to the project. Also, in each CSS
property, since the initial value specified by the file may contain variables, different units
or calculation functions which need to be normalised into a final value, called the computed value.
An example would be \texttt{calc(45deg - 1turn)} being serialised to \texttt{-315deg}.
The way normalisation should be done is not precisely specified by the CSS specification, meaning
that different browser engines can differ to this regard by doing what is convenient for them. The
only guideline given by the specification is to try to normalise to a value as short as possible.

The third step is layout. The rendering engine computes the size and the position of each box
of the web page on the screen. Boxes can be elements in the HTML document, but also parts of those
elements such as lines of text or list markers. This part is not very relevant to the project,
since conic gradients don't affect the size or position of elements on the screen.

The fourth step is painting. Each box is painted using the styling and layout information computed
at previous steps. The painting is potentially done on different layers, making it possible to repaint
one layer without repainting other layers. 

The last step is rendering, which is essentially taking the layers from the previous step and
rendering them as one single image. Some CSS properties only are applied at this step, such as
\texttt{transform} (which performs translations, rotations and some other transforms),
or \texttt{animation}, since it is more convenient and performant for the browser engine to do so.

\section{The Gecko rendering engine}

In Gecko (the Firefox rendering engine), the CSS engine taking care of the styling step is almost
fully written in Rust, with some rare parts in C++. Rust is a low-level programming language
developed by Mozilla in 2010, although it has since developed its own independent community. Its
main goal is to address the shortcomings of C++ in terms of memory safety. It is mainly an imperative
programming language and provides features familiar to C developers such as structs or enums.
However, it also provides object-oriented programming features such as traits, which are similar
to interfaces. Functional programming features such as pattern matching or lambdas can also be found.

The layout code, although not too relevant, is almost 

% -----------------------------------------------------------------------------

\chapter{Implementation}
\label{chap:execution}

\vspace{1cm} 

\noindent



\section{Style System}


\subsection{Supporting angles in gradients}
\subsection{Parsing}

\section{Web Painting}

\subsection{Normalization}


\section{WebRender}

The main components 

\section{Skia}




\section{Web Platform Tests}











% This chapter is intended to describe what you did: the goal is to explain
% the main activity or activities, of any type, which constituted your work 
% during the project.  The content is highly topic-specific, but for many 
% projects it will make sense to split the chapter into two sections: one 
% will discuss the design of something (e.g., some hardware or software, or 
% an algorithm, or experiment), including any rationale or decisions made, 
% and the other will discuss how this design was realised via some form of 
% implementation.  

% This is, of course, far from ideal for {\em many} project topics.  Some
% situations which clearly require a different approach include:

% \begin{itemize}
% \item In a project where asymptotic analysis of some algorithm is the goal,
%       there is no real ``design and implementation'' in a traditional sense
%       even though the activity of analysis is clearly within the remit of
%       this chapter.
% \item In a project where analysis of some results is as major, or a more
%       major goal than the implementation that produced them, it might be
%       sensible to merge this chapter with the next one: the main activity 
%       is such that discussion of the results cannot be viewed separately.
% \end{itemize}

% \noindent
% Note that it is common to include evidence of ``best practice'' project 
% management (e.g., use of version control, choice of programming language 
% and so on).  Rather than simply a rote list, make sure any such content 
% is useful and/or informative in some way: for example, if there was a 
% decision to be made then explain the trade-offs and implications 
% involved.


% \section{Example Section}

% This is an example section; 
% the following content is auto-generated dummy text.
% \lipsum

% \subsection{Example Sub-section}

% \begin{figure}[t]
% \centering
% foo
% \caption{This is an example figure.}
% \label{fig}
% \end{figure}

% \begin{table}[t]
% \centering
% \begin{tabular}{|cc|c|}
% \hline
% foo      & bar      & baz      \\
% \hline
% $0     $ & $0     $ & $0     $ \\
% $1     $ & $1     $ & $1     $ \\
% $\vdots$ & $\vdots$ & $\vdots$ \\
% $9     $ & $9     $ & $9     $ \\
% \hline
% \end{tabular}
% \caption{This is an example table.}
% \label{tab}
% \end{table}

% \begin{algorithm}[t]
% \For{$i=0$ {\bf upto} $n$}{
%   $t_i \leftarrow 0$\;
% }
% \caption{This is an example algorithm.}
% \label{alg}
% \end{algorithm}

% \begin{lstlisting}[float={t},caption={This is an example listing.},label={lst},language=C]
% for( i = 0; i < n; i++ ) {
%   t[ i ] = 0;
% }
% \end{lstlisting}

% This is an example sub-section;
% the following content is auto-generated dummy text.
% Notice the examples in Figure~\ref{fig}, Table~\ref{tab}, Algorithm~\ref{alg}
% and Listing~\ref{lst}.
% \lipsum

% \subsubsection{Example Sub-sub-section}

% This is an example sub-sub-section;
% the following content is auto-generated dummy text.
% \lipsum

% \paragraph{Example paragraph.}

% This is an example paragraph; note the trailing full-stop in the title,
% which is intended to ensure it does not run into the text.

% -----------------------------------------------------------------------------

% \chapter{Critical Evaluation}
% \label{chap:evaluation}

% {\bf A topic-specific chapter, of roughly $15$ pages} 
% \vspace{1cm} 

% \noindent
% This chapter is intended to evaluate what you did.  The content is highly 
% topic-specific, but for many projects will have flavours of the following:

% \begin{enumerate}
% \item functional  testing, including analysis and explanation of failure 
%       cases,
% \item behavioural testing, often including analysis of any results that 
%       draw some form of conclusion wrt. the aims and objectives,
%       and
% \item evaluation of options and decisions within the project, and/or a
%       comparison with alternatives.
% \end{enumerate}

% \noindent
% This chapter often acts to differentiate project quality: even if the work
% completed is of a high technical quality, critical yet objective evaluation 
% and comparison of the outcomes is crucial.  In essence, the reader wants to
% learn something, so the worst examples amount to simple statements of fact 
% (e.g., ``graph X shows the result is Y''); the best examples are analytical 
% and exploratory (e.g., ``graph X shows the result is Y, which means Z; this 
% contradicts [1], which may be because I use a different assumption'').  As 




% such, both positive {\em and} negative outcomes are valid {\em if} presented 
% in a suitable manner.

% -----------------------------------------------------------------------------

\chapter{Conclusion}
\label{chap:conclusion}

{\bf A compulsory chapter,     of roughly $5$ pages} 
\vspace{1cm} 

\noindent
The concluding chapter of a dissertation is often underutilised because it 
is too often left too close to the deadline: it is important to allocation
enough attention.  Ideally, the chapter will consist of three parts:

\begin{enumerate}
\item (Re)summarise the main contributions and achievements, in essence
      summing up the content.
\item Clearly state the current project status (e.g., ``X is working, Y 
      is not'') and evaluate what has been achieved with respect to the 
      initial aims and objectives (e.g., ``I completed aim X outlined 
      previously, the evidence for this is within Chapter Y'').  There 
      is no problem including aims which were not completed, but it is 
      important to evaluate and/or justify why this is the case.
\item Outline any open problems or future plans.  Rather than treat this
      only as an exercise in what you {\em could} have done given more 
      time, try to focus on any unexplored options or interesting outcomes
      (e.g., ``my experiment for X gave counter-intuitive results, this 
      could be because Y and would form an interesting area for further 
      study'' or ``users found feature Z of my software difficult to use,
      which is obvious in hindsight but not during at design stage; to 
      resolve this, I could clearly apply the technique of Smith [7]'').
\end{enumerate}

\section{}

% =============================================================================

% Finally, after the main matter, the back matter is specified.  This is
% typically populated with just the bibliography.  LaTeX deals with these
% in one of two ways, namely
%
% - inline, which roughly means the author specifies entries using the 
%   \bibitem macro and typesets them manually, or
% - using BiBTeX, which means entries are contained in a separate file
%   (which is essentially a databased) then inported; this is the 
%   approach used below, with the databased being dissertation.bib.
%
% Either way, the each entry has a key (or identifier) which can be used
% in the main matter to cite it, e.g., \cite{X}, \cite[Chapter 2}{Y}.

\backmatter

\bibliography{dissertation}

% -----------------------------------------------------------------------------

% The dissertation concludes with a set of (optional) appendicies; these are 
% the same as chapters in a sense, but once signaled as being appendicies via
% the associated macro, LaTeX manages them appropriatly.

% \appendix

% \chapter{An Example Appendix}
% \label{appx:example}

% =============================================================================

\end{document}
